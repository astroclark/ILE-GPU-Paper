% $Id$
\documentclass[twocolumn,prd,nofootinbib]{revtex4}


\newcommand\ForInternalReference[1]{}
%\newcommand\SkipForEarlyCirculation[1]{}
\newcommand\SkipForEarlyCirculation[1]{#1}
%\newcommand\SkipBoxing[1]{}
\newcommand\SkipBoxing[1]{#1}
\newcommand\SkipPP[1]{}
\usepackage{verbatim}
\usepackage{graphicx}
\usepackage{dcolumn}
\usepackage{bm}
\usepackage{color}
\usepackage{xspace}
\usepackage{url}
\usepackage{amsmath}
%\usepackage{adjustbox}
\usepackage{float}
\usepackage{multirow}
%
%
\usepackage{times}
%
%
%
\newcommand\optional[1]{}

%
\newcommand\E[1]{\left\langle #1\right\rangle}
\newcommand\qmstate[1]{\left|#1\right \rangle}
\newcommand\qmstateKet[1]{\left\langle#1\right|}
\newcommand\qmstateproduct[2]{\left\langle#1|#2\right\rangle}
\newcommand\qmoperatorelement[3]{\left\langle#1\left|#2\right|#3\right\rangle}
\newcommand\qmoperator[1]{{\bf #1}}
%
\newcommand\Y[1]{{{}_{#1}Y}}

\newcommand\lnL{ \ln {\cal L}}
\newcommand\lnLmarg{ \ln{\cal L}_{\rm marg}{}}
\newcommand\unit[1]{{\rm #1}}

\newcommand\rapidPEOrig{rapid\_pe1}
\newcommand\ILE{ILE}
\newcommand\editremark[1]{{\color{red} #1}}
%
%
%
\usepackage{color}
\definecolor{amber}{rgb}{1.0, 0.75, 0.0}
\definecolor{orange}{rgb}{1.0, 0.5, 0.0}
\definecolor{amaranth}{rgb}{0.9, 0.17, 0.31}
\def\fixme#1{\textcolor{red}{#1}}
\newcommand{\Richard}[1]{ {\color{blue}{#1}}}
\newcommand{\ros}[1]{ {\color{blue}{#1}}}
%

%

%
%
%
%
\graphicspath{{./figures/}}
\newcommand{\mc}{{\cal M}}
\newcommand{\Ms}{M_{\odot}}
\newcommand\LambdaTilde{\widetilde{\Lambda}}
\newcommand\DeltaLambdaTilde{\delta \widetilde{\Lambda}}
%
\def\ltsima{$\; \buildrel < \over \sim \;$}
\def\simlt{\lower.5ex\hbox{\ltsima}}
\def\gtsima{$\; \buildrel > \over \sim \;$}
\def\simgt{\lower.5ex\hbox{\gtsima}}

\newcommand\prx{Phys.~Rev.~X}
\def\aj{Astronomical Journal}                 %
\def\apj{Astrophysical Journal}                %
\def\apjl{Astrophysical Journal}             %
\def\pasj{PASJ}
\def\apjs{ApJS}              %
\def\mnras{MNRAS}            %
\def\prd{Phys.~Rev.~D}       %
\def\prl{Phys.~Rev.~Lett}    %
\def\cqg{Class.~Quant.~Grav.~}%
\def\araa{ARA\&A}             %
\def\nat{Nature}              %
\def\aap{A\&A}                %
\def\aapr{A\&A~Rev.~}    %
\def\pasp{PASP}    %
\def\sovast{Soviet Ast.}
%
%

\newcommand{\IMRPD}{\textsc{IMRPhenomD}\xspace}
\newcommand{\IMRPDT}{\textsc{IMRPhenomD\_NRTidal}\xspace}
\newcommand{\IMRP}{\textsc{IMRPhenomPv2}\xspace}
\newcommand{\SEOBP}{\textsc{SEOBNRv3}\xspace}
\newcommand{\SEOBA}{\textsc{SEOBNRv4}\xspace}
\newcommand{\SEOBAROM}{\textsc{SEOBNRv4\_ROM}\xspace}
\newcommand{\NRSur}{NRSur7dq2\xspace}
\newcommand{\TEOB}{SEOBNRv4T\xspace}
\newcommand{\Resum}{TEOBResumS\xspace}
\newcommand\RIFT{RIFT}
\newcommand{\Taylor}{TaylorF2\xspace}
\newcommand\PaperDetection{\underline{LVC-detect}\cite{DiscoveryPaper}}
\newcommand\PaperPE{\underline{LVC-PE}\cite{PEPaper}}
\newcommand\PaperTestGR{\underline{LVC-TestGR}\cite{TestingGRPaper}}
\newcommand\PaperPENRMethods{\underline{PE+NR-Methods}\cite{gwastro-PENR-Methods-Lange}}
\newcommand\PaperAstro{\underline{LVC-Astro}\cite{AstroPaper}}
\newcommand\PaperBurst{\underline{LVC-Burst}\cite{BurstPaper}}
\newcommand\PaperRates{\underline{LVC-Rates}\cite{RatesPaper}}
\newcommand\PaperStochastic{\underline{LVC-Stochastic}}
\newcommand\PaperSEOBNRvthree{\underline{LVC-SEOBNRv3}\cite{SEOBv3Paper}}

\def\RIT{Center for Computational Relativity and Gravitation, Rochester Institute of Technology, Rochester, New York 14623, USA}

\begin{document}

\title{Accelerating parameter inference with graphics processing units}
\author{D. Wysocki}
\affiliation{\RIT}
\author{Yao-Lung L. Fang}
%\author{Yao-Lung L. Fang (方耀龍)}
\affiliation{Computational Science Initiative, Brookhaven National Laboratory, Upton, NY 11973, USA}
\affiliation{National Synchrotron Light Source II, Brookhaven National Laboratory, Upton, NY 11973, USA}
\author{R. O'Shaughnessy}
\affiliation{\RIT}
\author{Jake - if he does runs}
\affiliation{\RIT}
\begin{abstract}
Gravitational wave Bayesian parameter inference involves repeated comparisons of GW data to generic candidate predictions.
Even with algorithmically efficient methods like RIFT or reduced-order quadrature, the time needed to perform these
calculations and overall computational cost can be significant compared to the minutes to hours needed to achieve the
goals of low-latency multimessenger astronomy.  By translating some elements of the RIFT algorithm to operate on
graphics processing units (GPU), we demonstrate substantial performance improvements, enabling dramatically reduced
overall cost and latency. 
\end{abstract}
\maketitle

\section{Introduction}
The Advanced LIGO  \cite{2015CQGra..32g4001T}  and Virgo \cite{gw-detectors-Virgo-original-preferred}  ground-based gravitational wave (GW) detectors have
identified several coalescing compact binaries
\cite{DiscoveryPaper,LIGO-O1-BBH,2017PhRvL.118v1101A,LIGO-GW170814,LIGO-GW170608,LIGO-GW170817-bns}.  
%
The properties of the sources responsible have been inferred via Bayesian comparison of  each source with candidate
gravitational ave signals
\cite{DiscoveryPaper,LIGO-O1-BBH,2017PhRvL.118v1101A,LIGO-GW170814,LIGO-GW170608,LIGO-GW170817-bns}
\editremark{citations: real events, also LI and RIFT} \cite{gwastro-PE-AlternativeArchitectures,gwastro-PENR-RIFT}.  
Many more are expected as observatories reach design sensitivity \cite{2016LRR....19....1A}.  
Both to handle the rapid pace of discovery and particularly to enable coordinated multimessenger followup, GW
observatories should be able to reconstruct the full source parameters of coalescing binaries as fast as possible
\cite{LIGO-whitepaper-2015} \editremark{update}.    
However, particularly when using the best-available waveform models, these calculations can be very costly.  In
practice, detailed inferences even using simplified waveforms take weeks for binary neutron stars, and can take months
for costly time-domain effective-one-body models which incorporate the effects of precession.


% BORROWED FROM ROM: edit
Several strategies have been
developed to reduce the 
%overall computational cost and turnaround time of 
computational cost of parameter estimation \cite{gwastro-pe-Brandon-STF2,gwastro-PE-AlternativeArchitectures,gw-astro-ReducedOrderQuadraturePE-TiglioEtAl2014,2016PhRvD..94d4031S,2017arXiv170302062V}.  
% The computational cost of parameter estimation can be reduced 
Approaches that have appeared in the literature include
%by 
generating the approximate solutions more quickly \cite{gwastro-mergers-PE-ReducedOrder-2013,2014PhRvX...4c1006F,2013PhRvD..87l2002S,2013PhRvD..87d4008C,gwastro-mergers-IMRPhenomP,gwastro-SpinTaylorF2-2013}; 
%by 
interpolating some combination
of the waveform or likelihood \cite{gwastro-approx-ROMNR-Blackman2015,2013PhRvD..87d4008C,2013PhRvD..87l2002S,2014CQGra..31s5010P,2014PhRvD..90d4074S,gwastro-PE-AlternativeArchitectures,cole2014likelihood,2012MNRAS.421..169G};  or 
%by 
adopting a sparse representation to reduce the computational cost of data
handling \cite{antil2013two,gwastro-mergers-PE-ReducedOrder-2013,2016PhRvD..94d4031S,gw-astro-ReducedOrderQuadraturePE-TiglioEtAl2014,gwastro-PE-AlternativeArchitectures}.   % 2014PhRvX...4c1006F,2013PhRvD..87l2002S,2013PhRvD..87d4008C
 Some methods, however, achieve rapid turnaround through simplifying approximations.  
%
The RIFT strategy described in \editremark{citations - include ROM} eschews these simplifications, performing
embarassingly-parallel inferences even for costly models.   However, the method as previously implemented still had a
significant net  computational cost and noticable startup time, limiting its diverse potential applications.

In this paper we describe a substantial improvement to the RIFT approach, enabled by translating its low-level
algorithms to operate on graphics processing units (GPUs).


This paper is organized as follows.
In Section \ref{Methods}, we review the underlying marginalized likelihood calculations used by RIFT; their updated GPU
implementation; and the operating settings which enable substantially increased performance.
In Section \ref{sec:demo}, we quantify the improved performance of our  GPU-accelerated code, and its agreement with the
original CPU-only version.


\section{Methods}

\subsection{Parameter inference with the RIFT likelihood}
% From RIFT paper
\ILE{}  -- a specific algorithm to ``integrate over extrinsic parameters'' -- provides a straightforward and efficient mechanism to compare any specific candidate gravitational wave source with
real or synthetic data   \cite{gwastro-PE-AlternativeArchitectures,NRPaper,2017PhRvD..96j4041L,2017CQGra..34n4002O},
by marginalizing the likelihood of the data over the seven coordinates characterizing the spacetime coordinates and
orientation of the binary relative to the earth.  
Specifically the likelihood of the data given gaussian noise has the form  (up to normalization)
\begin{equation}
\label{eq:lnL}
\ln {\cal L}(\bm{\lambda} ;\theta )=-\frac{1}{2}\sum\limits_{k}\langle h_{k}(\bm{\lambda} ,\theta )-d_{k} |h_{k}(\bm{\lambda} ,\theta )-d_{k}\rangle _{k}-\langle d_{k}|d_{k}\rangle _{k},
\end{equation}
where $h_{k}$ are the predicted response of the k$^{th}$ detector due to a source with parameters ($\bm{\lambda}$, $\theta$) and
$d_{k}$ are the detector data in each instrument k; $\bm{\lambda}$ denotes the combination of redshifted mass $M_{z}$ and the
remaining parameters needed to uniquely specify the binary's dynamics; $\theta$ represents the
seven extrinsic parameters (4 spacetime coordinates for the coalescence event and 3 Euler angles for the binary's
orientation relative to the Earth); and $\langle a|b\rangle_{k}\equiv
\int_{-\infty}^{\infty}2df\tilde{a}(f)^{*}\tilde{b}(f)/S_{h,k}(|f|)$ is an inner product implied by the k$^{th}$ detector's
noise power spectrum $S_{h,k}(f)$. 
%In all calculations, we adopt the fiducial O1 noise power spectra associated with data near GW150914 \cite{DiscoveryPaper}.
In practice we adopt both  low- and high- frequency cutoffs $f_{\rm max},f_{\rm min}$ so all inner products are modified to
\begin{equation}
\label{eq:overlap}
\langle a|b\rangle_{k}\equiv 2 \int_{|f|>f_{\rm min},|f|<f_{\rm max}}df\frac{[\tilde{a}(f)]^{*}\tilde{b}(f)}{S_{h,k}(|f|)}.
\end{equation}
The joint posterior probability of $\bm{\lambda} ,\theta$ follows from Bayes' theorem:
\begin{equation}
p_{\rm post}(\bm{\lambda} ,\theta)=\frac{ {\cal L}(\bm{\lambda} ,\theta)p(\theta)p(\bm{\lambda})}{\int d\bm{\lambda} d\theta {\cal L}(\bm{\lambda} ,\theta)p(\bm{\lambda})p(\theta)},
\end{equation}
where $p(\theta)$ and $p(\bm{\lambda})$ are priors on the (independent) variables $\theta ,\bm{\lambda}$. For each $\bm{\lambda}$, we evaluate the marginalized likelihood
\begin{equation}
 {\cal L}_{\rm marg}\equiv\int  {\cal L}(\bm{\lambda} ,\theta )p(\theta )d\theta
\end{equation}
via direct Monte Carlo integration, where $p(\theta)$ is uniform in 4-volume and source orientation.  
The integral in time is performed at a low level, by direct quadrature. 
For the remaining dimensions, to evaluate the likelihood in regions of high importance, we use an adaptive Monte Carlo as described in
\cite{gwastro-PE-AlternativeArchitectures}.    As described in Pankow et al, this marginalized likelihood can be evaluated efficiently
because, having generated the dynamics and outgoing radiation in all possible directions once and for all for fixed
$\mathbf{\lambda}$, the likelihood can be evaluated as a function of $\theta$ at very low computational cost.


% From ROM paper
A dimensionless, complex gravitational-wave
strain
%% ~\footnote{\scott{\sout{We use a distance-independent dimensionless strain $R h/M$, where
%% $R$ is the distance from the binary's center-of-mass
%% and $M$ is the total mass. In this paper we choose units so that $c=G=1$.}
%% [Remove this? I think it conflicts with other conventions/definitions 
%% used throughout the paper.]}}
\begin{align} \label{eq:strain}
h(t,\vartheta,\phi;\lambda) =  h_+(t,\vartheta,\phi;\lambda) - 
                                i h_\times (t,\vartheta,\phi;\lambda) \, ,
\end{align}
can be expressed in terms of its two fundamental polarizations $h_+$ and $h_\times$.
Here, $t$ denotes time, $\vartheta$ and $\phi$ are the polar and azimuthal angles
for the direction of gravitational wave propagation away from the source. 
The complex gravitational-wave strain can be written in terms of
spin-weighted spherical harmonics $\Y{-2}_{\ell m} \left(\vartheta, \phi \right)$ as 
\begin{align} \label{eq:strain_mode}
h(t,\vartheta,\phi;\lambda) = 
\sum_{\ell=2}^{\infty} \sum_{m=-\ell}^{\ell} \frac{D_{\rm ref}}{D} h^{\ell m}(t;\lambda) \Y{-2}_{\ell m} \left(\vartheta, \phi \right) \, ,
\end{align}
where the sum includes all available harmonic modes $h^{\ell m}(t;\pmb{\lambda})$ made available by the model;  where
$D_{\rm ref}$ is a fiducial reference distance; and where $D$, the luminosity distance to the  source, is one of the
extrinsic parameters.  

Following \citet{gwastro-PE-AlternativeArchitectures}, we substitute expression~\eqref{eq:strain_mode} 
for $h_{\ell m}$ into the expression $h_k(t) =F_{+,k} h_+(t_k) +
  F_{\times,k}h_\times(t_k)$ for the detector response $h_k$, 
where $t_k=t_c - \vec{x}_k \cdot \hat{n}$ is the arrival time at the $k$th detector (at position $\vec{x}_k$)
for a plane wave propagating along $\hat{n}$ \cite{gwastro-PE-AlternativeArchitectures}.
We then substitute these expressions for $h_k$ into the likelihood function~\eqref{eq:loglikelihood}
thereby generating~\cite{gwastro-PE-AlternativeArchitectures}
\begin{widetext}
\begin{align}
\ln {\cal L}(\lambda, \theta) 
&= (D_{\rm ref}/D) \text{Re} \sum_k \sum_{\ell m}(F_k \Y{-2}_{\ell m})^* Q_{k,lm}(\lambda,t_k)\nonumber \\
&   -\frac{(D_{\rm ref}/D)^2}{4}\sum_k \sum_{\ell m \ell' m'}
\left[
{
|F_k|^2 [\Y{-2}_{\ell m}]^*\Y{-2}_{\ell'm'} U_{k,\ell m,\ell' m'}(\lambda)
}
% \right. \nonumber \\ & \left.
 {
+  \text{Re} \left( F_k^2 \Y{-2}_{\ell m} \Y{-2}_{\ell'm'} V_{k,\ell m,\ell'm'} \right)
}
\right]
\label{eq:def:lnL:Decomposed}
\end{align}
\end{widetext}
%% \noindent \scott{[FOR RICHARD: is eq 9/10 missing objects? Looks like $t_c$ an $F_k$ just appeared. On the other hand,
%%     putting them into earlier equations will complicate the discussion and is awkward.]}  
where 
 where $F_k = F_{+,k} - i F_{\times,k}$ are the
complex-valued detector
response functions of the $k$th detector \cite{gwastro-PE-AlternativeArchitectures} and 
the quantities $Q,U,V$ depend on $h$ and the data as
\begin{subequations}
%\label{eq:ComputeRhoViaInnerProductMatrix}
\label{eq:QUV}
\begin{align}
Q_{k,\ell m}(\lambda,t_k) &\equiv \qmstateproduct{h_{\ell m}(\lambda,t_k)}{d}_k \nonumber\\
&= 2 \int_{|f|>f_{\rm low}} \frac{df}{S_{n,k}(|f|)} e^{2\pi i f t_k} \tilde{h}_{\ell m}^*(\lambda;f) \tilde{d}(f)\ , \\
{ U_{k,\ell m,\ell' m'}(\lambda)}& = \qmstateproduct{h_{\ell m}}{h_{\ell'm'}}_k\ , \\
V_{k,\ell m,\ell' m'}(\lambda)& = \qmstateproduct{h_{\ell m}^*}{h_{\ell'm'}}_k  \ .
\end{align}
\end{subequations}
Rewriting sums as matrix operations, the likelihood can be equivalently expressed as
\begin{align}
\ln {\cal L} &= \frac{D_{\rm ref}}{D} \text{Re}[ (F Y)^\dag Q]  \nonumber \\
 & - \frac{D_{\rm ref}^2}{4 D^2} [ (FY)^\dag U FY + (FY)^TV FY]
\label{eq:lnL:MatrixForm}
\end{align}
where $F,D$ are arrays over extrinsic parameters; $Q$ is an array over time and $(l,m)$; and $U,V$ ar matricies over
$(l,m)$. 



\subsection{Accelerated evaluation via efficient multiplication}
%* GPU code
%* Refactoring code to use the efficient GPU loops


At a fixed set of intrinsic parameters $\lambda$, \ILE{} will repeatedly evaluate the time-marginalized likelihood 
for each candidate Monte Carlo set of extrinsic parameters $\theta$:
\begin{eqnarray}
{\cal L}_{\rm margT} \equiv  \int {\cal L} \frac{dt}{T}
\label{eq:lnL:tmarg}
\end{eqnarray}
where $T$ is a small coincidence window consistent with the candidate event time; here and previously,
$T=0.3\unit{sec}$.   To accelerate the code, after precomputing the inner products $U,V,Q$, we simply shift them to the
graphics card, then carry out all calculations necessary to implement Eqs. (\ref{eq:lnL:MatrixForm}, \ref{eq:lnL:tmarg})
on the GPU.  To enable this implementation with portable code, we use \texttt{cupy}, a drop-in-replacement for equivalent
\texttt{numpy} code used for the CPU-based version of ILE.    For the most costly part of the calculation -- the inner
products necessary to evaluate $Q_{lm}(t)$ -- we use a custom CUDA kernel.  
%
With these changes alone, the likelihood evaluation is roughly $60\times$ faster on equivalent hardware.  
After this update, individual likelihood evaluation costs are not the performance- or cost-limiting feature of RIFT-based source
parameter inference.  Instead, the overhead associated with the adaptive Monte Carlo and with the (CPU-based) inner
product evaluations for $Q,U,V$ dominate our computational cost.


\subsection{Exploiting low-cost likelihoods}
Previously, each instance of ILE examined one intrinsic point $\lambda$.  The overall cost of this ILE evaluation
involved two parts: a startup cost, a setup cost, and a Monte Carlo integration cost.  The  startup cost $\tau_{start}$ is associated with code setup, followed by reading and Fourier-transforming the
data.  The setup cost $\tau_{setup}$ arises from waveform generation, followed by the inner products $U,V,Q$.  Finally, the Monte Carlo
cost $\tau_{mc}=N_{it} \tau_{it}$ increases linearly with the number of Monte Carlo iterations $N_{it}$. 
The choice of one $\lambda$ for each instance of ILE was eue to the substantial time  $\tau_{MC}$, which vastly
dominated the overall computational cost.   For example, for a typical analysis of a short signal --  a typical binary
black hole signal with $f_{\rm min}=20 \unit{Hz}$ and $m_1\simeq m_2\simeq 30 M_\odot$ --  this version
\editremark{cost table}, based on the assumption that the MC terminates after $N_{\rm it}\simeq 10^6$ iterations.  
By contrast, the marginalized likelihood described above  converges incredibly rapidly if the candidate model is
inconsistent with a signal (or the absence thereof) in the data, with only $N_{it} \simeq O(10^4)$  evaluations needed to identify the marginalized
likleihood.  


With the new low-cost likelihood evaluations, however, the startup and setup costs  $\tau_{start},\tau_{setup}$ now are
comparable or in excess of the Monte Carlo evaluation time.   For this reason, we reorganize the workflow, so each
instance of ILE loops over
$N_{eval}$ different choices for $\lambda$. 
%
Additionally, particularly for massive binary black holes, we adopt two additional performance improvements.  First and
foremost, we lower the sampling rate, thus lowering the startup cost $\tau_{start}$ and particularly setup cost
$\tau_{setup}$.  Previously and out of an abundance of caution, \ILE{} employed a sampling rate $1/\Delta t = 16384
\unit{Hz}$ for all calculations, but terminated all inner products in Eq. (\ref{eq:overlap}) at roughly $f_{\rm
  max}=2048\unit{Hz}$ or less.   Reducing the sampling rate by a factor $s$ decreases the  cost $\tau_{setup}$ by the
same factor. 
Also, to insure safe inner products over short data sets in the presence of very narrow spectral lines, we operated on a
significantly-padded data buffer,
using 
``inverse spectrum truncation'' \editremark{citation}.  For binary black holes, the degree of padding was significantly
in excess of the signal duration \editremark{numbers: say 64s for a BBH!}.  
Instead and following LI, we adopt a much shorter inverse spectrum truncation length, a much shorter padding buffer, and
a Tukey window applied to the data to remove discontinuities at the start and end of each segment.   
Reducing the duration of data analyzed by a factor $s'$ will reduce the cost $\tau_{setup}$ by a factor $s'$.  
%
Combining these factors together, the overall computational cost of each marginalized likelihood
evaluation $T_{ILE}/N_{eval}$  is
\begin{align}
T_{ILE}/N_{\rm eval} &= \tau_{start}/N_{\rm eval} + [ \tau_{setup}/ss' + N_{it} \tau_{it}] 
\end{align}
For a typical binary black hole signal with $f_{\rm min}=20 \unit{Hz}$ and $m_1\simeq m_2\simeq 30 M_\odot$, this
version requires $\tau_{start} \simeq 12\unit{s}$, $\tau_{setup}/ss' \simeq XX$, and $N_{it}\tau_{\it} \simeq XX$
\editremark{numbers}.
Typically we target $N_{\rm eval} \simeq 20$ so that startup costs are not a significant contributor to the overall
runtime.   In this configuration, the run time per marginalized likelihood is around $T_{ILE}/N_{\rm eval} \simeq 20
\unit{s}$ for a binary black hole \editremark{checkme - old number}.  


For low-latency analyses, we need to assess the overall wallclock time needed to perform an analysis, often with a
restricted number $N_{GPU}$ of available GPU-enabled machines.   To  completely $N_{\rm net}$ likelihood evaluations
with these resources requires
\begin{align}
T_{net} &= \frac{N_{\rm net}}{N_{\rm GPU} N_{\rm eval}} T_{ILE} \nonumber \\
& \simeq
  \frac{N_{net}}{N_{GPU}}\left[\tau_{start}/N_{\rm    eval} + [ \tau_{setup}/ss' + N_{it} \tau_{it}]  \right]
\end{align}
Typically we target $N_{\rm eval}  \lesssim 3\times 10^4, N_{\rm GPU} \simeq 100$, or roughly $300\times$ the cost per
individual marginalized likelihood evaluation.  For the binary black hole configuration described above, marginalized
likelihood evaluations will complete in about $100 \unit{minutes}$ \editremark{checkme}. This number can be reduced to
tens of minutes with a modestly larger GPU pool.
This discussion ignores the latency introduced by the  gaussian process interpolation stage at the end.  We will revisit
accelerated GP interpolation in subsequent work.


The \ILE{} likelhood  generates waveform dynamics and $h_{lm}(t)$ once per evaluation $\lambda$.   Waveform generation can contribute significantly to the
time needed to evaluate  $\ln {\cal L}_{\rm marg}$ for extremely costly
waveform models which require $\tau_{wf} $ a significant fraction of $  \tau_{setup}$.   \editremark{give examples}



\subsection{Tradeoff between Monte Carlo integration and accuracy}

Each marginalized likelihood evaluation has a relative uncertainty of order  $1/\sqrt{n_{\rm eff}}$, where the number of
effective samples $n_{\rm eff}=\epsilon N_{\rm it}$ increases linearly with the total number $N_{\rm it}$ of Monte Carlo
evaluations performed by \ILE{}.  Therefore, to increase (decrease) our accuracy for each likelihood evaluation by a factor of $A$
will require a factor $A^2$ more (fewer) iterations.   
%
We adopt a fixed threshold on $n_{\rm eff}$.  


We do benefit slightly by re-using the adaptive Monte Carlo intergrator for each extrinsic point $\lambda$.  Since the
integrator has already identified the likely range of sky locations and distances on the first iteration, each
subsequent evaluation of the marginalized likelihood can converge marginally more quickly.



\section{Validation and Performance}
\label{sec:demo}

\begin{figure}

\end{figure}

* Comparison: agreement.  Sky localization example

* Benchmarking performance (new speed for BBHs; also show BNS with tides)


\section{ Results (optional?)}

* Add Jake as author.  Rerun all O2 BBHs with SEOBNRv3?

\section{Conclusions}

* Enable low-latency parameter inference in basically real time for GW candidate sources, even for relatively costly
models with full physics


* Highly portable, generalizable to other codes: code will smoothly tranlate between GPU-only and CPU-only mode


* Not done yet: expect to transform MC and inner products, so all calculations remain on board, with just intermittent
control I/O from CPU


\begin{acknowledgements}
RO'S is supported by NSF PHY-1707965 and PHY-1607520.
DW thanks the RIT COS and CCRG for support, and  NSF-1707965.
Y-LLF was supported in part by BNL LDRD projects No.\,17-029 and No.\,19-002, 
and New York State Urban Development Corporation, d/b/a Empire State Development, under contract No.\,AA289.
We thank the Computational Science Initiative at BNL for hosting GPU Hackathon 2018, during which a part of this work was conducted. We also thank Jeff Layton at NVIDIA for bringing CuPy to our attention.
\end{acknowledgements}


%\bibstyle{unsrt}
\bibliography{paperexport}
\end{document}
